 \section{Introduction}
\lettrine{R}{educing} static stability of aircrafts is a way to increase flight performances in terms of manoeuvrability and drag reduction. The first case benefits fighter aircrafts which are usually designed unstable by manipulating stability surfaces and position of the center of gravity. When reducing the tail volums i.e stability surfaces and the level arms, we also achieve a significant reduction in drag due to the diminution of wetted surfaces.In the civilian domain, this effect is increaslingly exploited due to the important flight performances increase \cite{HermanImpactControlConceptonDesign}, \cite{Abzug}.

To obtain a high level of performance improvement, the aircraft design strategy must be modified so as to include a stability and control block that can act on the geometry of the airplane. This way, the designer can take advantage of fly by wire and active control. This domain is called Control Configured Vehicle (CCV) and is not limited to airplane\cite{Abzug}. An example of aircraft design based on CCV principles has been describe by Anderson and Mason in \cite{Anderson_CCV_design}. The authors introduce the main problematic with CCV; the automatisation of control law design and flight quality assessment in order to embbed the discipline in a Multi-Disciplinary Optimization MDO . They propose a solution based on fuzzy logic to include CCV in an MDO. In the following years B. Chudoba and H. Smith \cite{Chudoba_generic_method} as well as R. E. Perez and H. T. Liu \cite{LiuPerezMDOFramework}, both presented an MDO framework designing stability and control laws based on Stability Augmentation System and pole placement technics with B. Chudoba focusing on generic modeling and characterisation. More recently a study from \cite{WelsteadConceptualDesignAugmentedStability} shows a methodology for inclusion of stability and control using optimal control into MDO while Y. Denieul \cite{YannDenieul} shows in his thesis the integrated optimization of actuation surfaces together with the control law based on $H_\infty$ methods. All these examples tend towards a more global approach in aircraft design due to the coupling of two or more disciplines, here flight stability and control, flight performances.

While methods exist for designing aircrafts with relaxed longitudinal stability \cite{CosenzaHandlingQualities}, the reduction of lateral static stability is bounded by stability requirements arising from emergency situations . The vertical stabilizer is dimensioned to handle single engine failure and/or strong lateral winds up to 25 kt \cite{FeuersangerReducedStability}, \cite{AFC_NASA_report_Mooney}, \cite{NicolosiInvestigationVertical}, \cite{CS25}, \cite{MorrisFlDynCstinMDO}. Usually the parameter constraining the size of the vertical stabilizer is the Minimum Control Speed : $V_{MC}$ definied by certification specifications as the minimum velocity at which, if an engine is made inoperative, the vertical tail has sufficient control authority to counter the yawing moment created by the remaining engine at maximum take off power within $5\SIUnitSymbolDegree$ of bank angle \cite{CS25}. This constraint is a hard limit for the reduction of the rudder, impacting the design for the whole Vertical Tail (VT). In addition as Morris suggests in \cite{MorrisFlDynCstinMDO}, the VT is often over-dimensioned to cope with non-linear viscous effects such as masking by the fuselage. A lack of adapted tool able to predict the performance of the VT in these conditions is called responsible for this case.
%for the reduction of the vertical tail because it must satify a static equilibrium.

Efforts for reducing the fin are taken both in research institutions and aircraft industry. They include better performance prediction for design phase \cite{dellavecchia} as well as flow control to increase fin efficiency \cite{InnovativeFlow_Lin}. In this last example, a possible reduction of the fin area could amount to 15\% with a total drag reduction of 0.9\%.

We envision to increase this reduction by using a new game changing technology that has recently gained interest among aircraft propulsion system; Distributed Electric Propulsion and differential thrust. Differential thrust is already used in current multi-engine aircrafts in case of complete loss of hydraulic system as a mean to control the aircraft. In this case, it is called Propulsion Controlled Aircraft \cite{TouchdownPCA}, a system developed first at NASA Dryden center. However, it cannot be used as a mean to increase lateral stability because of the lack of engine redundancy and of the large reaction time of turbomachines.

Recently, more and more interest has been held on electric airplanes. Though studies usually focus on the challenge of energy storing, we will focus and exploit the characteristics of electric engines. These ones are our actuators for differential thrust. The reader is refered to \cite{MisconceptionMoore} for a complete list of electric engine advantages. We will retain only the following; \emph{scale free efficiency} and \emph{high power density}. They allow to rethink the way of airframe propulsion integration. The designer can freely place the engines where they may increase aerodynamic performances. Two effects are usually seeked; \textbf{1.}Increase maximum lift by blowing and \textbf{2.}Bounday Layer Ingestion for drag reduction \cite{Assessment_of_DEP}, \cite{Turboelec_prop_analysis_nasa}. This is the case for the all electric concept plane AMPERE from ONERA \cite{Ampere_concept} where efans are integrated to the wing to benefit from blowing effects and NASA aircraft X57\cite{DesignPerfSceptor}. For this last example the distributed engines are expected to replace high lift devices.
The advantages for lateral control with differential thrust are threefold:
\begin{itemize}
	\item \textbf{Redundancy} : if one or more critical engines are suddenly made inoperative, only a small portion of the thrust is lost. There remain an important number of engine to reallocate the thrust.
	\item \textbf{Reaction time}: with small electric engines the reaction time can be of the order of $10^{-1}$s \cite{ActionneurElectric}, therefore fast with respect to aircraft flight dynamics.
	\item \textbf{Level-arm}: the engines being distributed along the wing (in the previously mentioned airplanes), we beneficiate from an important level arm without adding specific structures.
\end{itemize}

%We further assume no aerodynamic interaction effects between the wing and the propulsion system. This is outside the scope of the present study. We only assume point force of direction always parallel to the body x-axis as shown in Fig~\ref{fig:AmpereTop}.

Though the idea has been suggested in \cite{Turboelec_prop_analysis_nasa}, example and study of differential thrust for yaw control and relaxed lateral stability has, to the best of the authors knowledge, not yet received much attention. Hence, the goal of our study is to explore the possibilities of relaxing lateral stability by using differential thrust with electric propulsion and more specifically the requirements in terms of design.
%The final delivery would be a block of stability and control ready to be included into an MDO which allows the designer to optimize the size of the vertical stabilizer and conceive the propulsion system accordingly.

A detailled treatment of the distributed propulsion is not the approach retained for this study. Because electric propulsion is easily integrated into airframe, important synergies can exit between aerodynamics, structure, flight control and propulsion. This multiphysic environment necessitate a global approach to study DEP aircraft. %The authors think that more practical and physical knowledge of flight control and flight qualities with differential thrust are needed.

Instead, a framework is proposed to assess different configurations of DEP aircrafts starting with a simple model of distributed propulsion. Only point force produced by engines evenly positionned on the wing are concidered for this study. Any interaction between slipstreams and airframe will be taken into account in a later stage once the framework is ready for complexifying the interactions.

A way of comparing the flight qualities of two different aircrafts has been proposed by Goman in \cite{GomanAttainableEqui} by computing and comparing the flight envelop of the two aircrafts. This method rely on finding equilibriums and allows to assess the stability of the aircraft. It is also possible to take into account a stability augmentation in the computation for an unstable aircraft. We selected this method and augmented it to take into account differential propulsion as well.

The present paper is organised as follow, first the mathematical treatment of the equilibrium search is performed with the distributed propulsion model. Then, it is explained how the aerodynamic database is constructed to study the variation of vertical tail area. Finally results obtained with this tool and the potential of differential thrust over traditional configuration are shown.

%The final paper will contain exploratory results, focusing on twin-engine turboprop regional transport aircrafts. We start by defining the requirements that DEP has to fulfill to maintain static equilibriums. Follows the exploration of dynamic properties achievable using simple stability augmentation methods and reflexion about overactuated system.
%
%All results will use theoretical and simulation tools. A model of the aircraft will be analyzed using vortex latice tools to obtain the aerodynamic coefficient with varying vertical tail surface area. Simulation will use 6 degrees non linear equations of motion to analyse the quality of the stability augmentation. In this extended abstract the analysis of equilibriums and sensibility parameters are presented.