%\documentclass[journal]{new-aiaa}
\documentclass[conf]{new-aiaa} %for conference papers
\usepackage[utf8]{inputenc}

\usepackage{graphicx}
\usepackage{subcaption}
\usepackage{amsmath}
\usepackage[version=4]{mhchem}
\usepackage{siunitx}
\usepackage{longtable,tabularx}
\usepackage{gensymb}
\usepackage{multirow}
\setlength\LTleft{0pt} 

\title{Towards an Aircraft with Reduced Lateral Static Stability Using Electric Differential Thrust}

\author{Eric Nguyen Van\footnote{PhD, eric.nguyen-van@isae.fr, eric.nguyen\_van@onera.fr}}
\affil{ISAE DCAS, Toulouse, France}
\affil{ONERA DTIS, Toulouse France}
\author{Daniel Alazard\footnote{Professor Researcher, daniel.alazard@isae.fr.} and Philippe Pastor\footnote{Professor Researcher, philippe.pastor@isae.fr.}}
\affil{ISAE DCAS, Toulouse, France}
\author{Carsten D\"oll\footnote{Research engineer, carsten.doll@onera.fr.}}
\affil{ONERA DTIS, Toulouse, France}
%
%\author{Eric Nguyen Van\footnote{PhD candidate, ISAE-ONERA}, second author\footnote{blablabla}}.
%\affil{ISAE-ONERA, Toulouse, France}
%\author{Philippe Pastor\footnote{Enseignant Chercheur, DCAS, philippe.pastor@isae.fr.} Daniel Alazard\footnote{Enseignant Chercheur, DCAS, daniel.alazard@isae.fr.}}
%\affil{ISAE-SUPAERO, Toulouse, France}
%\author{Carsten D\"\o ll \footnote{Research Engineer, DTIS, carsten.doll@onera.fr.}}
%\affil{ONERA, Toulouse, France}

%Title : "Towards an Aircraft with Reduced Static Stability using Electric Differential Thrust"
%
%Introduction :
%	Reduction of surface drag by reduction of stability surfaces
%	Problematic with the VT dimensioning
%	New technology : DEP
%	Study the possibilities of VT reduction by comparison of flight envelop
%Math modeling
%	Equation of flight
%	Modeling differential thrust
%	Solving the problem by optimization
%Parametric Aerodynamic Database
%	Panel/VLM analysis of ATR72 with and without VT
%	Introduction of Nicolossi's method to predict VT lift coefficient and correction of lateral coefficients
%Results (study of single/multiple engine failure at take off)
%	Baseline ATR72 (twin engine) with different VT size
%	DEP configuration of the Baseline with different VT size
%	Effect of varying the number of engine, ratio of inoperative engine and total on-board power
%Conclusion
%	Sum up results
%	Why differential propulsion is interesting to reduce VT
%	Implication and future work

\begin{document}

\maketitle

\begin{abstract}
In the context of aircraft drag reduction, we study the possibility of reducing the area of the vertical tail using Distributed Electric Propulsion while maintaining lateral stability with active Differential Thrust. Distributed Electric Propulsion is usually thought of as a mean to increase aerodynamic efficiency by exploiting the benefic effects of accelerating air around key parts of the aircraft. However, it can also be seen as a collection of actuation devices generating additional moments through Differential Thrust. When the engines are distributed along the lateral axis, the aircraft designer may take advantage of the increase of control authority on yaw to reduce the static stability or the control authority provided by the vertical tail. This in turn would allow a reduction of vertical tail surface area. In order to explore and assess this idea, we suggest a framework to compare flight qualities of a traditional configuration versus a configuration using Distributed Electric Propulsion and Differential Thrust. The framework provides information on the flight envelop and stability of the aircraft by computing a map of the equilibriums. Thanks to a global approach, it allows to study any aircraft or DEP configurations in any flight phase. In addition, a key feature of the framework is the inclusion of the VeDSC method to compute analytically the contribution of the vertical tail to lateral stability. It allows to study effects of a 30\% reduction of VT surface area. Here are presented the first results and potential of using differential thrust to reduce the area of the vertical tail and the reasons for us to continue developing this framework.
\end{abstract}

\clearpage

\section*{Nomenclature}

%\noindent(Nomenclature)

{\renewcommand\arraystretch{1.0}
\noindent\begin{longtable*}{@{}l @{\quad=\quad} l@{}}
CCV & Control Configured Vehicle\\
MDO & Multi-Disciplinary Optimization\\
DEP & Distributed Electric Propulsion\\
VT & Vertical Tail\\
AR & Aspect Ratio\\
$\rho$ & Air density (Kg/m$^3$)\\
$V$ & Speed (m/s)\\
$V_s$ & Stall speed (m/s)\\
$V_{MC}$ & Minimum control speed (m/s)\\
$P$ & Power (W)\\
$T$ & Thrust (N)\\
$\eta$ & Efficiency (-)\\
$S$ & Wing surface area (m$^2$)\\
$S_v$ & Vertical Tail surface area (m$^2$)\\
b & Wingspan (m)\\
$l_F$ & VT longitudinal position with respect to wing aerodynamic center (m)\\
$z_v$ & vertical position of the mean aerodynamic chord of the VT (m)\\
$V_v$ & VT volume (-)\\
$y$ & Lateral position of engine (m)\\
$m$ & Mass (Kg)\\
$g$ & Gravitational acceleration (m.s$^{-2}$)\\
$\beta$ & Side slipe angle ($\SIUnitSymbolDegree$)\\
$\phi$ & Bank angle ($\SIUnitSymbolDegree$)\\
$\gamma$ & Climb angle ($\SIUnitSymbolDegree$)\\
$\mu$ & Aerodynamic bank angle ($\SIUnitSymbolDegree$)\\
$\Omega$ & Turning rate (rad.s$^{-1}$)\\
$\delta_a$ & Aileron input ($\SIUnitSymbolDegree$)\\
$\delta_R$ & Rudder input ($\SIUnitSymbolDegree$)\\
$\delta_m$ & Throttle level (-)\\
$C_D$ & Drag force coefficient (-)\\
$C_{Y}$ & Lateral force coefficient (-)\\
$C_L$ & Lift force coefficient (-)\\
$C_l$ & Rolling moment coefficient (-)\\
$C_m$ & Pitch moment coefficient (-)\\
$C_n$ & Yawing moment coefficient (-)\\
$a_v$ & VT lift slope coefficient (-)\\
\multicolumn{2}{@{}l}{Subscripts}\\
v & Vertical Tail\\
0 & Nominal condition or sea level\\
b & Body attached frame\\
\end{longtable*}}


 \section{Introduction}
\lettrine{R}{educing} static stability of aircraft is a way to increase flight performances in terms of manoeuvrability and drag reduction. The first case benefits fighter aircraft which are usually designed unstable by manipulating stability surfaces and position of the center of gravity. When reducing the tail volums i.e stability surfaces and level arms, we also achieve a significant reduction in drag due to the diminution of wetted surfaces. In the civilian domain, this effect is increaslingly exploited due to the important flight performance increase \cite{HermanImpactControlConceptonDesign}, \cite{Abzug}.

To obtain a high level of performance improvement, the aircraft design strategy must be modified so as to include a stability and control block that can act on the geometry of the airplane. This way, the designer can take advantage of fly by wire and active control. This domain is called Control Configured Vehicle (CCV) and is not limited to airplane\cite{Abzug}. An example of aircraft design based on CCV principles has been describe by Anderson and Mason in \cite{Anderson_CCV_design}. The authors introduce the main problematic with CCV; the automatisation of control law design and flight quality assessment in order to embbed the discipline in a Multi-Disciplinary Optimization MDO . They propose a solution based on fuzzy logic to include CCV in an MDO. In the following years B. Chudoba and H. Smith \cite{Chudoba_generic_method} as well as R. E. Perez and H. T. Liu \cite{LiuPerezMDOFramework}, both presented an MDO framework with stability and control laws design based on stability augmentation system and pole placement technics, with B. Chudoba focusing on generic modeling and characterisation. More recently a study from \cite{WelsteadConceptualDesignAugmentedStability} shows a methodology for inclusion of stability and control using optimal control into MDO while Y. Denieul \cite{YannDenieul} shows in his thesis the integrated optimization of actuation surfaces together with the control law based on $H_\infty$ methods. All these examples tend towards a more global approach in aircraft design due to the coupling of two or more disciplines, here mainely flight performances and flight stability and control.

While methods exist for designing aircraft with relaxed longitudinal stability \cite{CosenzaHandlingQualities}, the reduction of lateral static stability is bounded by stability requirements arising from emergency situations . The vertical stabilizer is dimensioned to handle single engine failure and/or strong lateral winds up to 25 kt \cite{FeuersangerReducedStability}, \cite{AFC_NASA_report_Mooney}, \cite{NicolosiInvestigationVertical}, \cite{CS25}, \cite{MorrisFlDynCstinMDO}. Usually the parameter constraining the size of the vertical stabilizer is the Minimum Control Speed : $V_{MC}$ definied by certification specifications as the minimum velocity at which, if an engine is made inoperative, the vertical tail has sufficient control authority to counter the yawing moment created by the remaining engine at maximum take off power within $5\SIUnitSymbolDegree$ of bank angle \cite{CS25}. This constraint is a hard limit for the reduction of the rudder, impacting the design for the whole Vertical Tail (VT). In addition as Morris suggests in \cite{MorrisFlDynCstinMDO}, the VT is often over-dimensioned to cope with non-linear viscous effects such as masking by the fuselage. A lack of adapted tool able to predict the performance of the VT in these conditions is called responsible for this case.
%for the reduction of the vertical tail because it must satify a static equilibrium.

Efforts for reducing the fin are taken both in research institutions and aircraft industry. They include better performance prediction for design phase \cite{dellavecchia} as well as flow control to increase fin efficiency \cite{InnovativeFlow_Lin}. In this last example, a possible reduction of the fin area could amount to 15\% with a total drag reduction of 0.9\%.

We envision to increase this reduction by using a new game changing technology that has recently gained interest among aircraft propulsion system; Distributed Electric Propulsion and differential thrust. Differential thrust is already used in current multi-engine aircraft in case of complete loss of hydraulic system as a mean to control flight. In this case, it is called Propulsion Controlled Aircraft \cite{TouchdownPCA}, a system developed first at NASA Dryden center. However, it cannot be used as a mean to increase lateral stability because of the lack of engine redundancy and of the large reaction time of turbomachines.

Recently, more and more interest has been held on electric airplanes. Though studies usually focus on the challenge of energy storing, we will focus and exploit the characteristics of electric engines. These ones are our actuators for differential thrust. The reader is refered to \cite{MisconceptionMoore} for a complete list of electric engine advantages. We will retain only the following; \emph{scale free efficiency} and \emph{high power density}. They allow to rethink the way of airframe propulsion integration. The designer can freely place the engines where they may increase aerodynamic performances. Two effects are usually seeked: \textbf{1.}Increase maximum lift by blowing and \textbf{2.}Bounday Layer Ingestion for drag reduction \cite{Assessment_of_DEP}, \cite{Turboelec_prop_analysis_nasa}. This is the case for the all electric concept plane AMPERE from ONERA \cite{Ampere_concept} where efans are integrated to the wing to benefit from blowing effects and NASA aircraft X57\cite{DesignPerfSceptor}. For this last example the distributed engines are expected to replace high lift devices.
The advantages for lateral control with differential thrust are threefold:
\begin{itemize}
	\item \textbf{Redundancy} : if one or more critical engines are suddenly made inoperative, only a small portion of the thrust is lost. There remain an important number of engines to reallocate the thrust.
	\item \textbf{Reaction time}: with small electric engines the reaction time can be of the order of $10^{-1}$s \cite{ActionneurElectric}, therefore fast with respect to aircraft flight dynamics.
	\item \textbf{Airframe integration}: the engines being distributed along the wing (in the previously mentioned airplanes), we beneficiate from an important level arm without adding specific structures.
\end{itemize}

%We further assume no aerodynamic interaction effects between the wing and the propulsion system. This is outside the scope of the present study. We only assume point force of direction always parallel to the body x-axis as shown in Fig~\ref{fig:AmpereTop}.

Though the idea has been suggested in \cite{Turboelec_prop_analysis_nasa}, example and study of differential thrust for yaw control and relaxed lateral stability has, to the best of the authors knowledge, not yet received much attention. Hence, the goal of this study is to explore the possibilities of relaxing lateral stability by using differential thrust with electric propulsion and more specifically the requirements in terms of design.
%The final delivery would be a block of stability and control ready to be included into an MDO which allows the designer to optimize the size of the vertical stabilizer and conceive the propulsion system accordingly.

A detailled treatment of the distributed propulsion is not the approach retained for this study. Because electric propulsion is easily integrated into airframe, important synergies can exit between aerodynamics, structure, flight control and propulsion. This multiphysic environment necessitate a global approach to study DEP aircraft. %The authors think that more practical and physical knowledge of flight control and flight qualities with differential thrust are needed.

Instead, a framework is proposed to assess different configurations of DEP aircraft starting with a simple model of distributed propulsion. Only point force produced by engines equipped with propellers evenly positionned on the wing are concidered for this study. Any interaction between slipstreams and airframe will be taken into account at a later stage once the framework is ready for increasing complexity.

A way of comparing the flight qualities of two different aircraft has been proposed by Goman in \cite{GomanAttainableEqui} by computing and comparing the flight envelop of the two aircraft. This method relies on finding equilibriums and allows to assess the stability of an aircraft. It is also possible to take into account a stability augmentation in the computation for an unstable aircraft. We selected this method and augmented it to take into account differential propulsion as well.

The present paper is organised as follow, first the mathematical treatment of the equilibrium search is performed with the distributed propulsion model. Then, it is explained how the aerodynamic database is constructed to study the variation of vertical tail area. Finally results obtained with this tool and the potential of differential thrust over traditional configuration are shown.

%The final paper will contain exploratory results, focusing on twin-engine turboprop regional transport aircraft. We start by defining the requirements that DEP has to fulfill to maintain static equilibriums. Follows the exploration of dynamic properties achievable using simple stability augmentation methods and reflexion about overactuated system.
%
%All results will use theoretical and simulation tools. A model of the aircraft will be analyzed using vortex latice tools to obtain the aerodynamic coefficient with varying vertical tail surface area. Simulation will use 6 degrees non linear equations of motion to analyse the quality of the stability augmentation. In this extended abstract the analysis of equilibriums and sensibility parameters are presented.

%\clearpage

\section{Static equilibrium}
In this section we explain the mathematical fundamentals as well as the model used to capture differential thrust. The equation of flights in aerodynamic frame are first re-called, then modeling of propulsion is shown and finally the trimming algorithm is explained.

\subsection{Equation of flight}
The equations of flight in the aerodynamic frame, assuming uniform wind velocity are used. This form is the most commonly used when studying the lateral flight dynamics. Equations are taken from Boiffier \cite{Boiffier} and are equivalent to those presented by Goman in\cite{GomanAttainableEqui}:
\begin{align}
	\begin{pmatrix}
	m \dot{V} \\
	m\left[ \dot{\beta}V -V(p\sin\alpha - r\cos\alpha) \right]\\
	m\left[ \dot{\alpha} V \cos\beta + V\left(\sin\beta (p\cos\alpha + r\sin\alpha) - q\cos\beta\right)\right]
	\end{pmatrix}
	= & mg
	\begin{pmatrix}
	-\sin\gamma\\
	\cos\gamma \sin\mu\\
	\cos\gamma \cos\mu	
	\end{pmatrix}
	+ \frac{1}{2} \rho S V^2
	\begin{pmatrix}
	-C_D\\
	C_Y\\
	-C_L
	\end{pmatrix}
	+ \mathbf{H_{ab}} 
	\begin{pmatrix}
	F_{x_b}\\
	F_{y_b}\\
	F_{z_b}
	\end{pmatrix} \label{E:Sdtdebut}\\
	\textbf{I}
	\begin{pmatrix}
	\dot{p}\\
	\dot{q}\\
	\dot{r}
	\end{pmatrix}
	 + \begin{pmatrix}
	 p\\
	 q\\
	 r
	 \end{pmatrix} \times \textbf{I}
	 \begin{pmatrix}
	 p\\
	 q\\
	 r
	 \end{pmatrix}
 	 =& \frac{1}{2} \rho S V^2 l
	\begin{pmatrix}
	C_l\\
	C_m\\
	C_n
	\end{pmatrix}
	+
	\begin{pmatrix}
	M_{x_b}\\
	M_{y_b}\\
	M_{z_b}
	\end{pmatrix} \label{E:sdtMoments}
\end{align}

Where \textbf{I} is the inertia matrix, $\mathbf{H_{ab}}$ is the rotation matrix from body to aerodynamic frame projecting forces ($F_{x_b}$,$F_{y_b}$,$F_{z_b}$) due to propulsion:

\begin{align}
H_{ab} =& 
\begin{pmatrix}
\cos\alpha \cos\beta & sin\beta & \sin\alpha \cos\beta\\
-\cos\alpha \sin\beta & \cos\beta & \-sin\alpha \sin\beta\\
-\sin\alpha & 0 & \cos\alpha 
\end{pmatrix}
\end{align}

The weight is projected onto the aerodynamic airframe through the climb angle $\gamma$ and the aerodynamic bank angle $\mu$. These terms can be expressed as \cite{Boiffier}:
\begin{align}
\cos\gamma \sin\mu =& \sin\theta \cos\alpha \sin\beta + \cos\beta\cos\theta\sin\phi - \sin\alpha\sin\beta\cos\theta\cos\phi\\
\cos\gamma \cos\mu =& \sin\theta\sin\alpha + \cos\beta\cos\theta\cos\phi
\end{align}

The complementary kinematic equations are :

\begin{align}
\begin{pmatrix}
\dot{\phi}\\
\dot{\theta}\\
\dot{\psi}
\end{pmatrix}
= \begin{pmatrix}
1 & \sin\phi \tan\theta & \cos\phi \tan\theta\\
0 & \cos\phi & -\sin \phi\\
0 &\frac{\sin\phi}{\cos\theta} & \frac{\cos \phi}{\cos \theta}
\end{pmatrix}
\begin{pmatrix}
p\\
q\\
r
\end{pmatrix} \label{E:kinematics}
\end{align}

Additional parameters, namely the climb angle $\gamma$ and the turn rate $\Omega$ are definied as :
\begin{align}
\sin \gamma &= \cos\alpha\cos\beta\sin\theta - \sin\beta\sin\phi\cos\theta - \sin\alpha\cos\beta\cos\phi\cos\theta \label{E:gammaDef}\\
\Omega &= -\dot{\phi}\sin\theta + \dot{\psi} \label{E:OmegaDef}
\end{align}

The heading variable $\psi$ is of no interest for our study, in order to remove its contribution, we can replace the expression of $\dot{\phi}$ and $\dot{\psi}$ in equation (\ref{E:OmegaDef}) and obtain:
\begin{equation}
\Omega = \left( q \sin \phi + r \cos \phi \right) \sec \theta \label{E:OmegaUse}
\end{equation}

Leaving the heading angle aside, equations (\ref{E:Sdtdebut}), (\ref{E:sdtMoments}), (\ref{E:kinematics}), (\ref{E:gammaDef}) and (\ref{E:OmegaUse}) represent a set of $N_e=10$ equations to satisfy. The state vector is $\textbf{x}=[V,\alpha,\beta,p,q,r,\phi,\theta]$ and counts $n_x=8$ variables. The input vector corresponding to control surfaces, respectively ailerons, elevator and rudder is $\textbf{u}=[\delta_a, \delta_e, \delta_R]$ with $n_u=3$ variables. Finally we count $n_p=2$ additional parameters $\gamma$ and $\Omega$. The model is now ready to be complemented with a model of distributed propulsion.

\subsection{Propulsion modeling}

Propulsion and differential thrust are modeled together and are assumed to only contribute to generating forces and moments. Meaning that the rotor terms or gyroscopique effects are let aside such that equations of flight aren't further modified.
By summing the thrust forces and moments based on the geometrical arrangement shown in Fig~\ref{fig:AmpereTop} :
\begin{align}
\begin{pmatrix}
Fx_b\\
Fy_b\\
Fz_b
\end{pmatrix}
=
\begin{pmatrix}
\sum_{i=1}^{N} T_{x,i}\\
0\\
0
\end{pmatrix}
\text{ and }
\qquad
\begin{pmatrix}
Mx_b\\
My_b\\
Mz_b
\end{pmatrix}
=\begin{pmatrix}
0\\
0\\
\sum_{i=1}^{N} -T_{x,i} y_i
\end{pmatrix}
\end{align}

\begin{figure}[hbt!]
	\centering
	\includegraphics[width=.6\textwidth]{AmpereTop}
	\caption{Illustration of thrust distribution.} Point forces are considered symmetrically placed along the wing. Unlike the configuration of Ampere \cite{Ampere_concept}, equal repartition of engine from root to wing-tip is assumed. \label{fig:AmpereTop}
\end{figure}

Where $T_{x,i}$ and $y_i$ are respectively the thrust force and distance between the body X axis and the i$^{\text{th}}$ motor. 
Our study has previously been limited to propulsive system based on electric engine running propellers. For aircraft equipped with a propeller or a fan, a common thrust model can be \cite{SachsElectricPerf}:
\begin{equation}
	T=PV^{-1} \eta_p \frac{\rho}{\rho_0} \delta x \label{Eq:oldThrustModel}
\end{equation}
Where P is the engine power at sea level, V is the flight velocity, $\eta_p$ is the propeller or fan efficiency, $\rho$ the air density and $\delta_x$ the throttle command. Equation (\ref{Eq:oldThrustModel}) models the loss of power of air breathing engines with variation of air density. Electric motors on the contrary, do not suffer from rarefaction of air as turbomachines, such that we may concider the following thrust model for computing $T_{x,i}$ \cite{SachsElectricPerf}:
\begin{equation}
	T_{x,i}=\frac{P_E}{N}V^{-1}\eta_m\eta_p \delta_{x,i}
\end{equation}
With $P_E$ being the total electrical power available from the line, $N$ being the total number of engine, $\eta_m$ and $\eta_p$ respectively the engine and propeller efficiency (both concidered constant). Hence, we concider that the power is equally divided between each engine. Finally, $\delta_{x,i}$ is the throttle command of each engine and is added to the control input vector : $\textbf{u}=[\delta_a , \delta_e , \delta_R , \delta_{x,1} , \dots , \delta_{x,N}]$. The number of input becomes : $n_u=N_m+3$.

One may argue that electric propulsion will be impacted by altitude anyway. It is true in many aspects, for example:
\begin{itemize}
	\item With increasing altitude cooling of electric engine and power electronics can become more difficult
	\item Maximum rated voltage of conductors decreases with altitude due to Corona effect \cite{WiringSpace}
	\item Finally, in the case of series hybrid or turbo-electric propulsion, the turbo-machine producing the electric power remains sensible to air rarefaction.
\end{itemize}

These limitations are either due to technological locks that can be overcome with increasing interest in electric propulsion or associated with a level of details outside the scope of our study. For these reasons, we will keep the assumption that engine power is constant with altitude.

\subsection{Finding the trim position by optimization}

Similarly to what is done in \cite{GomanAttainableEqui}, a set of additional constraints $N_c$ can now be defined to condition the problem such that only one possible solution exists. The number of variables to determine ($n_x+n_u+n_p$) must equal the number of equations and constraints $N_e+N_c$.
In this case, considering that the number of engine varies, the number of additional constraints to define is given by:
\begin{align}
N_c=&n_x+n_u+n_p-N_e\\
N_c=&N_m+3
\end{align}

In the case where differential thrust is not used, the additional input is $N_m=1$, representing forward thrust or throttle level, and one must fix $N_c=4$ additional constraints, similarly as in \cite{GomanAttainableEqui}. These additional constraints are added to determine the flight condition, typically fixing the following variables: $[V,\beta,\gamma,\Omega]$. With differential propulsion, the minimum number of additional input is $N_m=2$ with two engines. Consequently, the problem becomes quickly overdetermined. An infinite number of equilibrium points can exist. This can be pictured by the different possible combination of throttle level and rudder action to satisfy a certain thrust and yaw moment.

For overdetermined problems, it is common to use optimization methods to find a satisfying solution \cite{OppeinheimerControlAllocation}. Two options are possible: one may define the input vector as $\mathbf{u}=[\delta_a,\delta_e,\delta_n, \delta_x]$ with $\delta_n$ being the total yaw moment input and $\delta_x$ the total thrust force such that the equilibrium problem is well conditioned and then use an optimization method to find the $\delta_R$ and $\delta_{x,i}$. Or one could run the optimization on the complete set of variables. We selected the second option for this study. This is motivated by the will of maintaining a tight coupling between flight dynamics and differential thrust.

It should be stressed that the first option leaves room for development of 
\begin{itemize}
	\item Generic control and allocation laws
	\item Optimal control
	\item Optimisation of design
\end{itemize}

These aspects will be the object of future studies.

Without loss of generality, additional higher and lower bounds are added on control inputs, angle of attack and bank angle depending on the flight phase. For example, the bank angle is limited to $\pm5\degree$ when studying engine failure at take off as stated by flight regulation \cite{CS25}. These bounds are resumed in table \ref{tab:Bonds} and are in part, dependent on the aircraft selected for the study.

The following variables are chosen to imposed the flight conditions $[V,\beta,\gamma,\Omega]$. The objective function to minimize is defined as the power required to maintain equilibrium. Such an objective function makes sense in the point of view of the designer who will look for minimizing the power to install on the aircraft. Finally, to simulate engine failure we simple add a constraint on the throttle level of the corresponding engines.

The problem hence writes:
\begin{align}
\underset{\tilde{x}}{min} & \: \:\sum_{i=1}^{N} T_{x,i} V\\
\text{With:}\qquad \tilde{x}&=[\alpha, p, q, r, \phi, \theta, \delta_a, \delta_e, \delta_R, \delta_{x,1}, \dots, \delta_{x,N}]\\
\text{Subjected to: }&\notag \\
\begin{pmatrix}
0 \\
-mV(p\sin\alpha - r\cos\alpha)\\
mV\left[\sin\beta (p\cos\alpha + r\sin\alpha) - q\cos\beta\right]
\end{pmatrix}
= & mg
\begin{pmatrix}
-\sin\gamma\\
\cos\gamma \sin\mu\\
\cos\gamma \cos\mu	
\end{pmatrix}
+ \frac{1}{2} \rho S V^2
\begin{pmatrix}
-C_d\\
C_Y\\
-C_L
\end{pmatrix}
+ \mathbf{H_{ab}} 
\begin{pmatrix}
\sum_{i=1}^{N} T_{x,i}\\
0\\
0
\end{pmatrix} \\
0 =& \frac{1}{2} \rho S V^2 l
\begin{pmatrix}
C_l\\
C_m\\
C_n
\end{pmatrix}
+
\begin{pmatrix}
0\\
0\\
\sum_{i=1}^{N} T_{x,i} y_i
\end{pmatrix} -
\begin{pmatrix}
p\\
q\\
r
\end{pmatrix}
\times \textbf{I}
\begin{pmatrix}
p\\
q\\
r
\end{pmatrix}\\
0= & p + q\sin\phi \tan\theta + r \cos\phi \tan\theta\\
0= & q\cos\phi -r\sin \phi\\
\Omega = & \left( q \sin \phi + r \cos \phi \right) \sec \theta \\
\sin \gamma = & \cos\alpha\cos\beta\sin\theta - \sin\beta\sin\phi\cos\theta - \sin\alpha\cos\beta\cos\phi\cos\theta\\
0 = & \delta_{x,1}\\
\vdots\notag\\
0 = & \delta_{x,j}
\end{align}

The problem being now defined, the next section will treat the constitution of the aerodynamic database to compute the aerodynamic coefficients in function of VT area as well as flight phase.

\begin{table}[hbt!]
	\caption{\label{tab:Bonds} Additional bounds depending on flight phase}
	\centering
	\begin{tabular}{l|c|c|c}
		Flight phase & $V<71 m.s^{-1}$& $V>71 m.s^{-1}$ & Engine failure\\
		\hline
		$\alpha (\degree)$ & $-2\leq\alpha\leq 10$ & $-2\leq\alpha\leq 15$ & (-) \\
		$\phi (\degree)$ & $\pm 30$ & $\pm 30$ & $\pm 5$\\ 
		$\theta (\degree)$ & $\pm 30$ & $\pm 30$& $\pm 30$\\
		$\delta_a(\degree)$ & $\pm 20$& $\pm 20$& $\pm 20$\\
		$\delta_e(\degree)$ & $\pm 20$ & $\pm 20$ & $\pm 20$\\
		$\delta_R(\degree)$ & $\pm 25$ & $\pm 25$ & $\pm 25$\\
		$\delta_{x,i}$ & $0< \delta_{x,i} \leq 1$ & $0< \delta_{x,i} \leq 1$ & $0< \delta_{x,i} \leq 1$ \\
	\end{tabular}
\end{table}



\clearpage

\section{Aerodynamic database}
In this section, it is explained how the aerodynamic database was obtained to study the effect of varying the vertical tail surface area. Assumptions are re-called to put the focus only on relaxed lateral stability.

\subsection{Reference Aircraft}
The method developed up to now is generic and could be used with any aircraft given the corresponding aerodynamic characteristics. To study the differences between a traditional configuration and a DEP aircraft, a baseline aircraft representing the class of aircraft most probable for electric propulsion and distributed propulsion has to be selected. Commuters aircraft are often cited as the next big step in developing electric airplanes since most of their mission are within the limits of electric propulsion in terms of endurance \cite{MisconceptionMoore} \cite{StucklMethodsDesignElectriProp}. These aircrafts usually are equipped with turboprop engine and fly at subsonic velocity. A good representative of this class of aircraft is the ATR72 which details are reported in table~\ref{tab:nominalset}.

\begin{table}[hbt!]
	\caption{\label{tab:nominalset} ATR 72 general details \cite{ATRFAAtypecertificate}, \cite{JanesAircraft}}
	\centering
	\begin{tabular}{l|c}
		Variables & Value\\
		\hline
		Wingspan & 27 m\\
		Wing surface area & 61 m$^2$\\ 
		VT surface area & $12$ $\textrm{m}^2$\\
		Engine level arm & 4.1 m\\
		Masse & 21500 Kg\\
		Total available power & 4 000 KW\\
		Stall velocity $V_s$ & 50.5 m/s\\
	\end{tabular}
\end{table}

So far, special care was given to keep the analysis as generic as possible to be able to compare flight qualities of different aircraft configurations. Following the same philosophy it is assumed no change in the geometry, mass or power available of the reference ATR72 between the baseline configuration and the electric propelled one.

The only feature allowed to change is the vertical tail which will be reduced so as to obtain relaxed lateral stability or a lightly unstable aircraft. Consequently, we will analyse a weakly unconventional aircraft configuration.

\subsection{Building the Aerodynamic Database}
For unconventional configurations, it is necessary to carefully select the method with which one can establish the aerodynamic database. As Chudoba explains in \cite{ChudobaUnconventionalConf}, the means of calculating aerodynamic characteristics can be organized in three categories summurised in Table~\ref{tab:aeroAnalysis}.
\begin{table}[hbt!]
	\caption{\label{tab:aeroAnalysis} Methods for Aerodynamic Analysis \cite{Chudoba_generic_method}}
	\centering
	\begin{tabular}{l|c}
		Category & Example of methods\\
		\hline
		Analytical & Lifting Line, Swept Wing Theory, ...\\
		Empirical, semi-empirical & DATCOM, ESDU, ...\\ 
		Numerical & Vortex Lattice Method, Panel Method, CFD, ...\\
	\end{tabular}
\end{table}

Both analytical and empirical/semi-empirical methods are built on experiences and analysis of conventional configurations. Therefore there can hardly be concidered as generic methods. Numerical methods offer different level of fidelity allowing to capture the specificity of unconventional design. For preliminary designs, VLM or Panel Method, both linear technics, are often prefered over CFD for their favorable precision relative to computational cost.

Although our configuration is only weakly unconventional, capturing the effect of geometrical changes in VT isn't something that analytical of empirical/semi-empirical methods can do because of the important influence of other aircraft components on the flow impacting the VT. This has been demonstrated by Nicolosi in \cite{NicolosiInvestigationVertical}, where his research team investigated the differences obtained between DATCOM, ESDU estimation technics and CFD methods complemented with wind tunnel experiments.

The main contributors to flow perturbation impacting the VT are the fuselage and the horizontal tail which are acting as end plates, reducing the downwash at the root and tip of the vertical tail. 

Using numerous CFD simulations complemented with wind tunnel test on a generic commuter aircraft model, Nicolosi and his research team could establish a new semi-empirical method called VeDSC. This method focuses on predicting the VT efficiency as a function of VT geometry and aircraft components. The main assumption is the fact that contribution of each components of the aircraft on lateral coefficients: \{$C_{Y}$, $C_{p}$, $C_{r}$\} $\equiv C_{\textrm{lat}}$, can be decoupled in the following way:
\begin{eqnarray}
C_{\textrm{lat}_\beta} = C_{\textrm{lat},F_\beta} + C_{\textrm{lat},W_\beta} + C_{\textrm{lat},v_\beta}
\end{eqnarray}
Where it is assumed that the contribution of the VT, $C_{\textrm{lat},v_\beta}$ is influenced by the fuselage, wing and horizontal tail but does not influence the other coefficients $C_{lat,F_\beta}$ and $C_{lat,W_\beta}$.
The VeDSC method furnishes a way to estimate the coefficient $C_{\textrm{lat},v_\beta}$ through a reformulation of the VT lift slope coefficient defined as follow:
\begin{equation}
a_v=K_F K_W K_H C_{L,v_\beta}
\end{equation}
Where subscripts F, W, v refer respectively to fuselage, wing and vertical tail. $C_{L,v_\beta}$ is the lift slope of a swept wing determined using Diederich formula for swept wing \cite{DiederichPlanformParameter}. $K_F$, $K_W$ and $K_H$ are corrective coefficients taking into account respectively the fuselage, wing and horizontal tail effects. The reader is refered to \cite{NicolosiVTDesignReview} and \cite{NicolosiDirectionalStabilityReviewofEmpiricalMethod} for the complete formulation of these parameters.

The VT contribution to lateral coefficients is then calculated using formulas given by Etkin \cite{Etkin}:
\begin{align}
&C_{Y_\beta} = a_v\frac{S_v}{S}\left( 1-\frac{\partial \sigma}{\partial \beta}\right) 
&&C_{l_\beta} =-a_v\frac{S_v z_v}{S b}\left( 1-\frac{\partial \sigma}{\partial \beta}\right) 
&&&C_{n_\beta} = a_v V_v\left( 1-\frac{\partial \sigma}{\partial \beta}\right)\\
&C_{Y_p} = -a_v\frac{S_v}{S}\left(2\frac{z_v}{b}-\frac{\partial \sigma}{\partial \hat{p}}\right) && \qquad &&&C_{n_p}= a_v V_v\left(2\frac{z_v}{b}-\frac{\partial \sigma}{\partial \hat{p}}\right)\\
&C_{Y_r} = a_v\frac{S_v}{S}\left( 2\frac{l_F}{b}+\frac{\partial \sigma}{\partial \hat{r}}\right)  &&C_{l_r} =a_v\frac{S_v z_v}{S b}\left( 2\frac{l_F}{b}+\frac{\partial \sigma}{\partial \hat{r}}\right) &&&C_{n_r} = -a_v V_v\left( 2\frac{l_v}{b}-\frac{\partial \sigma}{\partial \hat{r}}\right)
\end{align}

This semi-empirical methods required hundreds of CFD simulations to explore a wide variety of parameter changes. Variation and validity intervals of some parameters of interest are shown in Table~\ref{tab:VeDSCParam}. The method could be extrapolated to VT of aspect ratio from 0.5 to 4 since the Diederich formula is valid for these values however the correcting terms would be out of interpolation range.



Overall the VeDSC method turns to be an adequate semi-empirical method to be used with our reference aircraft. In addition, having a formula to estimate the VT efficiency instead of running a VLM is an important time saving aspect.

In order to establish our aerodynamic database, we modeled the ATR72 without its VT into VSPaero using publicly available data essentially from \cite{JanesAircraft}. The VLM included in VSPaero is then used to establish the longitudinal coefficients and contribution to the lateral coefficients. The contribution of the fuselage and wing to lateral coefficients has been found to be adequatly estimated by the VLM. It is then complemented with the VeDSC method to quickly and accurately account for the changes in VT geometry. This renders the database extremely agile since we only need to run VLM simulations once.

\begin{figure}[hbt!]
	\centering
	\begin{subfigure}[b]{0.33\textwidth}
		\includegraphics[width=1.0\textwidth]{CyCstAR}
		\caption{All derivatives per rad}
		\label{fig:CyCstAR}
	\end{subfigure}
	%add desired spacing between images, e. g. ~, \quad, \qquad, \hfill etc. 
	%(or a blank line to force the subfigure onto a new line)
	\begin{subfigure}[b]{0.33\textwidth}
		\includegraphics[width=1.0\textwidth]{ClCstAR}
		\caption{All derivatives per rad}
		\label{fig:ClCstAR}
	\end{subfigure}
	%add desired spacing between images, e. g. ~, \quad, \qquad, \hfill etc. 
	%(or a blank line to force the subfigure onto a new line)
	\begin{subfigure}[b]{0.33\textwidth}
		\includegraphics[width=1.0\textwidth]{CnCstAR}
		\caption{All derivatives per rad}
		\label{fig:CnCstAR}
	\end{subfigure}
	\caption{Evolution of lateral coefficients of total aircraft with variation of VT area for constant AR.} Whitest marker represents $S_v=0.1S_{v,0}$, darkest represents $S_v=S_{v,0}$, per step of $0.1S_{v,0}$.\label{fig:cstAR}
\end{figure}

To explore the flight performances in the whole flight envelop, it has been chosen to use a database in function of mach number as only flight parameter. For low speed flight, when the flaps are used, the parameters are assumed to be constant. Then, they vary with the mach number. The evolution of some coefficients with Mach number is illustrated in Fig~\ref{fig:MachVariation}. The Mach number 0.2 taken at see level represents the minimum flight velocity of the ATR72 without flaps.

\begin{figure}[hbt!]
	\centering
	\begin{subfigure}[b]{0.33\textwidth}
		\includegraphics[width=1.0\textwidth]{CyCstSpan}
		\caption{All derivatives per rad}
		\label{fig:CyCstSpan}
	\end{subfigure}
	%add desired spacing between images, e. g. ~, \quad, \qquad, \hfill etc. 
	%(or a blank line to force the subfigure onto a new line)
	\begin{subfigure}[b]{0.33\textwidth}
		\includegraphics[width=\textwidth]{ClCstSpan}
		\caption{All derivatives per rad}
		\label{fig:ClCstSpan}
	\end{subfigure}
	%add desired spacing between images, e. g. ~, \quad, \qquad, \hfill etc. 
	%(or a blank line to force the subfigure onto a new line)
	\begin{subfigure}[b]{0.33\textwidth}
		\includegraphics[width=1.0\textwidth]{CnCstSpan}
		\caption{All derivatives per rad}
		\label{fig:CnCstSpan}
	\end{subfigure}
	\caption{Evolution of lateral coefficients of total aircraft with variation of VT area for constant span.} Whitest marker represents $S_v=0.3S_{v,0}$, darkest represents $S_v=S_{v,0}$, per step of $0.1S_{v,0}$ The interval of AR swept is $[1.56,5.21]$\label{fig:cstSpan}
\end{figure}


\begin{figure}[hbt]
	\centering
	\begin{subfigure}[b]{0.49\textwidth}
		\includegraphics[width=0.8\textwidth]{CybetaMachChange}
		\caption{}
		\label{fig:CybetaMachChange}
	\end{subfigure}
	%add desired spacing between images, e. g. ~, \quad, \qquad, \hfill etc. 
	%(or a blank line to force the subfigure onto a new line)
	\begin{subfigure}[b]{0.49\textwidth}
		\includegraphics[width=0.8\textwidth]{CyrMachChange}
		\caption{}
		\label{fig:CyrMachChange}
	\end{subfigure}
	\\
	%add desired spacing between images, e. g. ~, \quad, \qquad, \hfill etc. 
	%(or a blank line to force the subfigure onto a new line)
	\begin{subfigure}[b]{0.49\textwidth}
		\includegraphics[width=0.8\textwidth]{CnbetaMachChange}
		\caption{}
		\label{fig:CnbetaMachChange}
	\end{subfigure}
	\begin{subfigure}[b]{0.49\textwidth}
		\includegraphics[width=0.8\textwidth]{CnrMachChange}
		\caption{}
		\label{fig:CnrMachChange}
	\end{subfigure}
	\caption{Evolution of lateral coefficients of total aircraft with Mach number.}\label{fig:MachVariation}
\end{figure}

Finally, two ways of modifying the vertical tail are available. The first is the change of surface area while maintaining the aspect ratio constant. This gives linearly varying coefficients as it can be seen in fig~\ref{fig:cstAR}. The second comes from the fact that one may want to keep the horizontal tail high to avoid the slipstream of the propellers, hence the surface should be reduced without modifying the span. In turn the aspect ratio is increased up to reasonnable values to limit extrapolation as shown in fig~\ref{fig:cstSpan}. This variation induces non linear variation of the coefficient and may be find useful for flight qualities.

\begin{table}[hbt]
	\caption{\label{tab:VeDSCParam} A few parameter ranges on which VeDSC has been constructed. From \cite{NicolosiNewApproach}} 
	\centering
	\begin{tabular}{l|l|c}
		Parameters & Description & Range\\
		\hline
		$A_v$ & VT aspect ratio & $\left[1,2\right]$\\
		$A_W$ & Wing aspect ratio & $\left[6,16\right]$\\
		$z_W$ & Wing vertical position, relative to fuselage centerline & $\left[-1,1\right]$ \\
		$z_H$ & Horizontal Tail Vertical Position, relative to VT span & $\left[0,1\right]$\\
		$\frac{S_H}{S_v}$ & Relative Horizontal tail surface area & $\left[0.5,2\right]$
	\end{tabular}
\end{table}

\clearpage

\section{Results} \label{Results}
The framework detailed before is used to generate the results presented in this section. First, the case study is explained as well as the format used to present the results. Then an interpretation of the results is proposed.

\subsection{Case study}
At this point, it is possible to study a wide variety of flight configuration and situations. To keep the study consice we will limit our analysis to the situation of engine failures at take off with landing gear retracted as described by CS25.121, CS25.147 and CS12.149 \cite{CS25}. These certification rules set the requirements on climb angle, directional control and $V_{MC}$ in case of one or multiple engine failures. Although this manner bounds the study to only one flight phase, a good insight is given on the tool efficiency to compare aircraft configuration and potential of differential thrust.

The certification rules indicate different flight conditions for twin engine and aircraft with more than four engines. The main differences are resumed in table~\ref{tab:DiffTwinMulti}. %CS25.121 states the following flight conditions:
\begin{table}[hbt]
	\caption{\label{tab:DiffTwinMulti} Certification requirements for twin-engines and more than four engines \cite{CS25}} 
	\centering
	\begin{tabular}{l|l|c}
		Parameters & Twin engine & More than four engines\\
		\hline
		Gradient of climb, critical engine inoperative & $\leq 2.4\%$ & $\leq 3.0\%$\\
		Heading change of $15\degree$ at 1.3$V_{SR_1}$& One engine inoperative & Two critical engines inoperative\\
		$V_{MC} < 1.13 V_{SR}$ & One engine inoperative & One critical engine inoperative \\
	\end{tabular}
\end{table}

It is of interest to investigate the most defavorable situation for differential thrust which is clearly at a $3\%$ climb angle and two or more critical engines inoperative. 

For aircraft with more than four engines, the certification specifies that sufficient control must remain possible with the two critical engines inoperative at $1.3V_{SR}$ to perform heading changes. While $V_{MC}$ has to be demonstrated with two engines inoperative only at landing where fuel shortage is expected. This is not as constraining as for the take off where a much higher power is needed. A carefully designed architecture would normally avoid the suddent stop of engines located one next to another as in the case of Ampere \cite{Ampere_concept}. Other risks however might damage closely located engines such as bird collision or rupture of a propeller.

For these reasons it has been decided to render at least three engines inoperative, representing one fourth of the total power. Thus the total number of engines for the ATR72 with DEP is fixed to twelve.

A total of four configurations are studied to capture the overall changes that differential thrust and VT reduction bring to the aircraft. These configurations are resumed in table~\ref{tab:ConfigurationStudied}. It must be stressed that velocities are calculated from publicly available data and thus are only indicative \cite{ATRspeed}.


\begin{table}[hbt]
	\caption{\label{tab:ConfigurationStudied} Aircraft Configurations}
	\centering
	\begin{tabular}{l|c|c|M{3cm}|M{3cm}}
		Configuration & Original & DEP 1 & DEP 2 & DEP 3\\
		\hline
		Description & Original ATR72 & DEP ATR72 & DEP ATR72 Differential Thrust & DEP ATR72 Differential Thrust and small VT\\
		\hline
		Engines & 2 & 12 & 12 & 12\\
		Inoperative engines & 1 & 3 & 3 & 3\\
		VT area & $S_{v_0}$ & $S_{v_0}$ & $S_{v_0}$ & $0.7 S_{v_0}$\\
		Rudder allowed & Yes & Yes & No & No\\
		Gradient of climb & $3.0\%$ &$3.0\%$ & $3.0\%$ & $3.0\%$\\
		Turn rate, $\Omega$ (rad/s)& 0 & 0& 0&0\\
		Propeller & \multicolumn{4}{c}{Feathered, no drag assumed}\\
		\hline
		\multicolumn{5}{c}{Additional parameters} \\
		\hline
		$V_{\textrm{app}}$=$1.13V_{\textrm{app}}$ (m/s)& \multicolumn{4}{c}{56}\\
		$V_{\textrm{sr}}$ (m/s)& \multicolumn{4}{c}{50.5}\\
		$1.3V_{\textrm{sr}}$ (m/s)& \multicolumn{4}{c}{65}\\
		VT stall limit & \multicolumn{4}{c}{$\beta=15\degree$}\\
		\hline
		\multicolumn{5}{c}{Flight parameters for throttle repartition and rudder deflection}\\
		\hline
		$\beta$ & \multicolumn{4}{c}{$0\degree$}\\
		V(m/s) & \multicolumn{4}{c}{60}\\
	\end{tabular}
\end{table}

The flight conditions represent a rectilinear climb at constant velocity. The same climb angle is imposed for all configuration to allow comparison in an identical flight situation.
For each configuration an equilibrium map is built by swapping side slip angle and flight velocity. Along the flight envelop comes the rudder deflection and throttle repartition of a single equilibrium point. Side slip angle and velocity for this point is shown in table~\ref{tab:ConfigurationStudied}

%\begin{itemize}
%\item Climb gradient with one engine inoperative may not be less than 2.4\% for twin engines
%\item Climb gradient with one engine inoperative may not be less than 3\% for aircraft with four engines or more.
%\end{itemize}
\clearpage
\subsection{Flight envelop map}
\subsubsection{Map Interpretation}\label{MapInterpretation}
The flight envelops along with the accompaning throttle repartition and rudder deflection are shown in Fig~\ref{MapOrignialTwin+DEP} to Fig~\ref{fig:DEPfin07_15enginesMap+Defl}. For each equilibrium map, a point indicates an equilibrium. If this equilibrium is on the edge of the stability map a line shows the limiting parameter. It is either the $5\degree$ limitation in roll, stall or rudder saturation.
For distributed propulsion, engine saturation is indicated with different markers. A rectangular marker signifies that one engine is saturated, up-triangle two engines, down triangle three engines, left triangle four engines and finally right triangle five engines.
Additionally, the complete zone after $\|\beta\|\geq 15\degree$ is faded, signifying that any equilibrium is valid under the condition that the VT did not yet experienced stall.

\subsubsection{ATR72 twin engine and Distributed propulsion}

The baseline ATR72 is presented in Fig~\ref{fig:originalfin1_3engine} and the trim inputs in Fig~\ref{fig:Defloriginalfin1_3engine}.
The climb gradient is slightly unfavorable for this configuration nevertheless, one can observe the compliance with the controlability requirement at low velocity and the $15\degree$ margin at $1.3V_{SR}$. The limiting parameters for the flight envelop are the roll angle limitation for negative side slipe, rudder deflection for positive side slipe. The map is not symmetrical as expected for the case of engine failure.

\begin{figure}[hbt]
	\centering
	\begin{subfigure}{0.49\textwidth}
		\includegraphics[width=0.95\textwidth]{originalMapBetaVelfin1Eng3RudFalse}
		\caption{Original ATR72, flight envelop.}
		\label{fig:originalfin1_3engine}
	\end{subfigure}
	\begin{subfigure}{0.49\textwidth}
		\includegraphics[width=0.95\textwidth]{Defloriginalfin1Eng3RudFalse}
		\caption{Original ATR72, throttle level and rudder deflection.}
		\label{fig:Defloriginalfin1_3engine}
	\end{subfigure}
	\caption{Map of equilibrium points of the Original configuration and throttle level and rudder deflection for trim at V=60m/s, $\beta=0$} \label{MapOrignialTwin+DEP}
\end{figure}

Fig~\ref{fig:originalfin1_15engine} and Fig~\ref{fig:Defloriginalfin1_15engine} show the same airplane equipped with a distributed propulsion however not using differential thrust as shown in Fig~\ref{fig:Defloriginalfin1_15engine}. Despite the fact that the power loss is only one fourth of the total power, rudder deflection remains similar to the previous case and the equilibrium map is only increased by a few degres to the right.

This can be reasonnably explained by the fact that the outter most engines on the left wing, despite representing only one fourth of the power, have a high level arm thus creating an important yaw moment. At this stage, designing an aircraft with Distributed Propulsion would not bring substantial reduction on the VT surface area since it remains the primary control effector for yaw. This changes once Differential Propulsion is activated.

\begin{figure}
	\begin{subfigure}{0.49\textwidth}
		\includegraphics[width=0.95\textwidth]{originalMapBetaVelfin1Eng15RudFalse}
		\caption{DEP 1 flight envelop.}
		\label{fig:originalfin1_15engine}
	\end{subfigure}
	\begin{subfigure}{0.49\textwidth}
		\includegraphics[width=0.95\textwidth]{Defloriginalfin1Eng15RudFalse}
		\caption{DEP 1, throttle level and rudder deflection.}
		\label{fig:Defloriginalfin1_15engine}
	\end{subfigure}
	\caption{Map of equilibrium points of the configuration DEP 1 and throttle level and rudder deflection for trim at V=60m/s, $\beta=0$} \label{DeflOrignialeNoDiffThrust}
\end{figure}

\subsubsection{ATR72 using differential thrust and small VT}
Fig~\ref{fig:DEPoriginalfin1_15engine} and Fig~\ref{fig:DeflDEPoriginalfin1_15Eng} respectively show the equilibrium map and the trim inputs for an ATR72 with distributed propulsion and differential thrust activated.

It is important to remember that to segregate the possible increase of control provided by the differential thrust, the rudder isn't activated for the remaing configurations as it is shown in Fig~\ref{fig:DeflDEPoriginalfin1_15Eng}. Continuing on the same figure, the thrust repartition found by the optimizer is linear except for the saturated engine(s) as it is the case here.

The equilibrium map shows one main characteristic intrinct to differential thrust: control power decreases with increasing velocity.

In consequence, since the flight enveloped is increased at low velocity with respect to Fig~\ref{fig:originalfin1_3engine} and Fig~\ref{fig:originalfin1_15engine}, it can be stated that differential thrust is able to maintain lateral control at low velocity. However at $1.3V_{SR}$ the control margin is too low.

\begin{figure}[hbt]
		\centering
		\begin{subfigure}{0.49\textwidth}
			\includegraphics[width=0.95\textwidth]{DEPoriginalMapBetaVelfin1Eng15RudTrue}
			\caption{DEP 2, flight envelop.}
			\label{fig:DEPoriginalfin1_15engine}
		\end{subfigure}
		\begin{subfigure}{0.49\textwidth}
			\includegraphics[width=0.95\textwidth]{DeflDEPoriginalfin1Eng15RudTrue}
			\caption{DEP 2, throttle level and Rudder deflection at 60m/s and $\beta=0\degree$.}
			\label{fig:DeflDEPoriginalfin1_15Eng}
		\end{subfigure}
		\caption{Configuration DEP 2 using only Differential Thrust. Rectangle indicate an equilibrium with one engine at saturation (full throttle), for map interpretation see paragraph \ref{MapInterpretation}}\label{DEPfin1_15engMap+Defl}
\end{figure}

The main reason why the flight envelop is reduced at high velocity is because the VT is generating a restoring moment as soon as side slip appears. To increase the flight envelop, one simply has to reduced the VT, which is performed in Fig~\ref{fig:DEPoriginalfin07_15engine} and Fig~\ref{fig:DeflDEPoriginalfin07_15Eng}. Here the VT surface area is reduced by $30\%$, the aircraft remains positively damped, that is $C_{n_\beta}>0$, however the equilibrium map improves vastly. The directional control margin at $1.3V_{SR}$ is now compliant with the certification despite the fact that a higher number of engines are being used at full throttle to maintain equilibrium.


\begin{figure}[hbt!]
	\centering
	\begin{subfigure}{0.49\textwidth}
		\includegraphics[width=0.95\textwidth]{DEPoriginalMapBetaVelfin07Eng15RudTrue}
		\caption{DEP 3, flight envelop}
		\label{fig:DEPoriginalfin07_15engine}
	\end{subfigure}
	\begin{subfigure}{0.49\textwidth}
		\includegraphics[width=0.95\textwidth]{DeflDEPoriginalfin07Eng15RudTrue}
		\caption{DEP 3, throttle level and rudder deflection at 60m/s and $\beta=0\degree$}
		\label{fig:DeflDEPoriginalfin07_15Eng}
	\end{subfigure}
	\caption{DEP 3, with $S_v=0.7S_{v_0}$. Rectangle indicates an equilibrium with one engine at saturation (full throttle), for map interpretation see paragraph \ref{MapInterpretation}}\label{fig:DEPfin07_15enginesMap+Defl}
\end{figure}

\subsubsection{Interpretation and limits of the results}
So far the following observation were made:
\begin{itemize}
	\item Equipping an ATR72 with distributed propulsion but not differential thrust would not guarranty a reduction of VT surface area due to the important level arm of engines located at the wing tip.
	\item Differential thrust alone would be sufficient to maintain full control of the aircraft at low velocity due to the high control effectiveness.
	\item The VT actually has to be reduced to be able to exploit the full potential of differential thrust.
\end{itemize}
\vspace{0.25cm}
Limitations can be anticipated if one extrapolates these results. The reduction of VT surface area could be further increased but then stability at high velocity, when differential thrust loses its control effectiveness would be badly degraded. This is observable as each aircraft employing differential thrust see their equilibrium map reducing with increasing velocity. Nevertheless this shows that employing differential thrust brings a new paradigm for the design of the VT by changing the design case.

The study is limited to the precision of the thrust model employed as well as the important assumption that no interaction takes place between propeller and wing. Models taking these effects into account can now be added to the framework.
The framework itself suffers from the underconstrained system of equations mostly when differential thrust and rudder are allowed together. It seems that the objective function doesn't induce enough constraints on the repartition of the yaw moment. The definition of an explicit allocation function could further increase the capability of the framework to predict trim points with differential thrust. Additionally, this study is limited by the aerodynamic models that consider non linear effects only with VT geometry and linear effects otherwise. That is no consideration of stall, efficiency decrease with rudder deflection or masking of VT by the fuselage.

As future work, implementation of interaction models and more precise thrust model has to be performed along with a better formulation for yawing moment distribution. In parallel of such development, the framework allows to study the dynamic behaviour of such configurations with or without stability augmentation system. It is also necessary to assess the variation of stability gains observed with the configuration i.d. the total number of engines, the ratio of inoperational engines as well as the total on board power. These will be part of future research in an effort to integrate this work in an multidisciplinary optimisation process for preliminary sizing.


%----- Study Case -----
%One original ATR72 at take off without failed engine next to the same ATR72 at Take off with SEF:
%	Shows that Vmc is found for take off,
%	Show that control is maintained at 1.3Vsr (=15° yaw toward inoperative engine)
%	shows that no equilibrium exists,
%	eventually show the degradation of equilibrium with reduced VT
%	Show the engine saturation at high speed

%Show the same ATR with distributed propulsion:
%	Show input histogramme with and without rudder
%	Show map without rudder and with small rudder
%	Show map with small fin and rudder
%
%At take off means :
%	3% slope for more than 4 engine condition, 
%	2.4 % for twin engine with landing gear retracted (CS.25.121)
%	Not necessarly full power but explore by going higher (in speed mostly)
%	0 altitude
%Maybe replace velocity scale by Vsr, 1.3Vsr etc... as defined by CS

% ATR at 21.5T Vapp=56m/s if it is 1.13Vsr then Vsr=49.55m/s, 1.3Vsr=64.4m/s
% Eventually look at the 20° turns requirement from the certif

%
%Determine 1.3V_{sr} for CS25.147 stating 15° yaw in the direction of the inoperative engine.
%STATE MINIMUM DRAG and 

%\clearpage 


\section{Conclusion}
Reduction of Vertical Tail surface area is tackled in this study considering an emerging and game changing technology: Differential Thrust with Distributed Electric Propulsion (DEP). This technology could benefit commuter aircrafts as they are envisioned to be the first all electric aircrafts for passenger transportation. To assess the flight stability and subsequent handling qualities, a framework relying on a trimming algorithm has been developed allowing to compare a tradditional aircraft with one using differential thrust. To take into account the effect of the reduction of Vertical Tail, we could rely on the VeDSC method developed at Naples University. This method increases largely the precision of previous preliminary aerodynamic performance calculation methods such as DATCOM or ESDU by interpolating results of hundreds of high fidelity CFD simulations. Non linear effects parametrized by geometrical parameters are captured and renders our framework extremely versatil to study aircrafts of the type of the ATR72. A simplified formulation of distributed propulsion, without propeller wing interaction effect is used in order to quickly estimate the potential of differential thrust to control the aircraft. 

Results show an important increase of control at low velocity, where propellers are generating the highest thrust while control at high velocity is limited. Interestingly, reducing the Vertical Tail conducts to a dramatic increase of the flight envelop, while relying exclusively on Differential Thrust. Overall, it has been shown that using differential thrust can result in reduction of Vertical Tail as well as a new paradigm for the design of the Vertical Tail. This one would have to be sized for a different flight case most probably at higher velocity, assuring a better efficiency of the Vertical Tail and hence a substantial reduction.
Further studies will concern the variation of the performances with the configuration (number of engines, number of inoperational engines etc...) such as to guide the designer who whishes to design an aircraft using Differential Thrust as well as the introduction of a propeller-wing interaction model.

%We are studying the possibility of reducing lateral static stability of an aircraft using Distributed Electric Propulsion and differential thrust. Attention was brought onto the advantages and current difficulties of reducing the area of the fin. Thanks to a new propulsion technology that will be tested in flight in the following years, we suggested that using differential thrust in combination with electric engines is a reliable way to greatly reduced the vertical stabilizer.
%The potential of this idea is studied by looking at static equilibriums in critical phases of engine failure at take off, which is the dimensioning case for vertical stabilizer.
%Preliminary results show sensibility analysis and good performances. Reduction of total installed power seems feasible as well as excellent redundancy depending on total power installed. Following results will include parametric modelisation of the aircraft and aerodynamic simulation tools to better capture the reduction of vertical stabilizer on the aerodynamic coefficients. Dynamic behaviour with basic control law and behaviour at high velocity will also be investigated.

%\clearpage
%\section{Structure}
%
%\section{Introduction}
%\subsection{Aircraft design, stability requirements}
%\begin{itemize}
%	\item Laterale stability is fixed primarily by certification requirements \textbf{cite EASA CS, thèse Feuersanger, sweeping jets others}.
%	\item Name quantities used to qualify performances $V_{mc}$, extreme conditions.
%\end{itemize}
%
%\subsection{Reduction of stability, history}
%\begin{itemize}
%	\item Name aircrafts flying with reduced longitudinal static stability, and name means of stabilization (\textbf{cite abzug}). 
%	\item Introduce the new design schem and CCV domain (\textbf{Find more litterature and aircraft examples})
%\end{itemize}
%
%
%\subsection{Reduction of stability on laterale axis}
%\begin{itemize}
%	\item Name work performed on laterale axis
%	\item Cite the difficulties encountered for this axis due to stability and control requirements in emergency situations. 
%	\item Cite precedent work on reduction of the VT by rendering it more efficient using sweeping jet (\textbf{cite sweeping jets}) both as a mean and potential gains
%	\item cite augmented CCV NASA study
%\end{itemize}
%
%
%\subsection{Differential thrust}
%
%\begin{itemize}
%	\item Introduce electric propelled airplanes and design changes concerning this technology, in particular efficency versus size, possibilities of having distributed engines to take advantages of aerodynamic blowing \textbf{cite NASA, ampere}.
%	\item This study aims at exploring the possibilities offered by such a configuration.
%\end{itemize}
%
%
%\subsection{Structure of the study}
%\begin{itemize}
%	\item What do we aim -> regional transport aircraft which are the first forseen to be manufactured
%	\item How do we contribute -> explore faisability, sensibility parameters such as power to install, configurations and control strategy to the destination of the designer. Ready a brick to include to multi-disciplinary optimization
%	\item Mean -> simulation study, construction of the simulator and design loop with quick and low fidelity aero simulation tools. Control loops tested versus actuation performances (thrust delivered, position of thrust and reaction time).
%\end{itemize}
%
%\section{Preliminary results}
%Based on experimentally determined coefficients of a Beech99 non-linear motion model. Lateral coefficients being artifically reduced proportionally to the surface of the VT.
%\begin{itemize}
%	\item \textbf{Static requirements} from static equilibrium.
%	\item Identification of the problem, over determined system. \textbf{cite overdetermined systems} application of different methodology.
%	\item Dynamic requirements
%\end{itemize}
%
%
%\section{Work remaining}
%Increase fidelity using more elaborated aero tools and a parametric representation of the aircraft in order to determine through simulation aero coefficients of the aircraft with and without VT.
%Complete sensibility analysis.
%Explore the limitations at larger velocity.
%Explore dynamic behaviour and control allocation strategy in linear and non-linear simulations.



\section*{Acknowledgments}
The author would like to thank Mamoun El Oueldrhiri for his support in the exploration of equilibriums with distributed propulsion.

\bibliography{sample}

\end{document}
